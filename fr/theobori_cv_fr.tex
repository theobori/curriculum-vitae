\let\nofiles\relax % This is because res says to not emit aux files,
                   % but lastpage needs aux files.
\documentclass[margin,line]{../res}
\usepackage{hyperref}
\usepackage[utf8]{inputenc}
\usepackage[T1]{fontenc}
\usepackage{microtype}
\usepackage{fourier-orns}
\usepackage{amsmath,amssymb}
\usepackage{lastpage}
\usepackage{fancyhdr}
\usepackage{etaremune}
\usepackage[normalem]{ulem}
\usepackage[style=iso]{datetime2}

\oddsidemargin -.5in
\evensidemargin -.5in
\voffset -25pt
%\topmargin -.2in
\headsep 25pt
\textwidth=6.0in
\textheight=8.9in
\itemsep=0in
\parsep=0in

% Headings
\pagestyle{fancy}
\lhead{Théo BORI --- Curriculum Vitae}
\chead{}
\rhead{\thepage\ sur \pageref*{LastPage}}
\lfoot{}
\cfoot{}
\rfoot{}
\renewcommand{\headrulewidth}{0.4pt}
%\renewcommand{\footrulewidth}{0.4pt}

% Give hyperref some metadata
\hypersetup{pdftitle={Théo BORI — Curriculum Vitae (CV)},
  pdfauthor={Théo BORI},
  pdfsubject={Théo BORI's curriculum vitae (CV)},
  pdfkeywords={informatique, unix, linux, système, ingénierie logicielle,
    SRE, DevOps, sysadmin, développement},
  colorlinks=true,
    linkcolor=blue,
    filecolor=magenta,
    urlcolor=blue
}

\begin{document}

\newcommand{\myname}{Théo BORI}
\newlength{\mynamewidth}
\settowidth{\mynamewidth}{\namefont\myname}

\name{\hspace*{0.5\textwidth}\hspace{-0.5\mynamewidth} \myname \vspace*{.1in}}
% On the first page, have no header.
\thispagestyle{empty}

\begin{resume}

	\section{\sc Informations de contact}
	%\vspace{.05in}
\href{mailto:theo1.bori@epitech.eu}{theo1.bori@epitech.eu} \hfill
\href{https://theobori.cafe}{Website} \href{https://www.github.com/theobori}{GitHub} \href{https://www.linkedin.com/in/theo-bori}{LinkedIn}\\



	% Added to improve page breaks
	\vspace{-1em}

	\section{\sc Profil}
	Étudiant de 24 ans en 5\textsuperscript{ème} année à \href{https://www.epitech.eu/}{Epitech}, passionné par les systèmes UNIX, Linux et le monde open-source. J'ai réalisé ma 4\textsuperscript{ème} année à \href{https://www.tudublin.ie/}{Technological University Dublin} où j'ai obtenu un double diplôme en data science.

	\section{\sc Expérience profesionnelle}
	 {\bf Site Reliability Engineer}, \href{https://www.vinci-autoroutes.com}{VINCI Autoroutes}, Vedène, France
	\hfill {\it Septembre 2024--Mars 2024}
	\vspace*{.05in}
	\begin{itemize}
		\item Création de ressources Azure suivie de la mise en place de Backstage (AKS).
		\item Instanciation d'une infrastructure, sécurisation des accèss et intégrations des services de l'usine logicielle à Backstage.
		\item \'Ecriture d'un automatisme pour intégrer les projets existants dans la forge logicielle à Backstage.
	\end{itemize}

	{\bf Site Reliability Engineer}, \href{https://www.vinci-autoroutes.com}{VINCI Autoroutes}, Vedène, France
	\hfill {\it Avril 2023--Juillet 2023}
	\vspace*{.05in}
	\begin{itemize}
		\item Déploiement automatisé des ressources dans Azure, de NeuVector (AKS) et de sa configuration avec Terraform.
		\item Contributions au code source de NeuVector, création d'un \href{https://github.com/theobori/go-neuvector}{SDK}, d'un \href{https://github.com/theobori/terraform-provider-neuvector}{provider Terraform} et mise en place de sa supervision.
		\item Documentation complète de la mise en oeuvre de NeuVector.
	\end{itemize}

	{\bf Développeur web et mobile}, \href{https://skwad.com/}{Skwad}, Montpellier, France
	\hfill {\it Juillet 2021--Décembre 2021}
	\vspace*{.05in}
	\begin{itemize}
		\item Maintenance et amélioration continue des applications, correction de bugs, ajout de fonctionnalités et gestion de bases de données relationnelles.
		\item Développement d'une application mobile, comprenant l'ajout de fonctionnalités, la sécurisation des serveurs, création d'un backend (API REST) et la conteneurisation des applications.
	\end{itemize}

	\section{\sc Expérience personnelle}

	\href{https://github.com/theobori/tinyfilter}{Filtrage de paquets avec Linux}
	\begin{itemize}
		\item Utilisation de la bibliothèque Linux eBPF qui permet de charger une application dans l'espace noyau.
	\end{itemize}

	Interpréteurs Lox
	\begin{itemize}
		\item Réalisation d'interpréteurs avec différentes méthodes d'analyse syntaxique et d'évaluation de code, dont un \href{https://github.com/theobori/tinylox}{interpréteur Tree-Walk} et une \href{https://github.com/theobori/lox-virtual-machine}{machine virtuelle de bytecodes}.
	\end{itemize}


	\href{https://github.com/theobori/tinychip}{Émulateur de CHIP-8}
	\begin{itemize}
		\item Interpréteur du langage CHIP-8 avec une émulation vidéo et audio.
	\end{itemize}

	Paquetage d'applications
	\begin{itemize}
		\item \href{https://github.com/theobori/openbsd-ports}{Portages} d'applications et de jeux compatibles avec X11 sur OpenBSD.
		\item \href{https://repology.org/maintainer/theo1.bori@epitech.eu}{Contributions} à la collection de paquets Nix et \href{https://github.com/theobori/nix-teeworlds}{modularisation} de service.
	\end{itemize}

	\href{https://github.com/theobori-cafe}{Site web et services}
	\begin{itemize}
		\item \href{https://theobori.cafe}{Site web} sur lequel je partage les choses que je trouve intéressantes, je mets aussi à disposition des \href{https://services.theobori.cafe}{services} gratuits et ouverts à toutes et à tous.
	\end{itemize}

	\section{\sc Compétences informatiques}
	\textbf{Langages}: Go, Rust, TypeScript, C++/C, Python, shells UNIX, Nix(OS)\\
	\textbf{Autres}: Azure, Kubernetes, Docker, Terraform, Prometheus, Grafana

\end{resume}
\end{document}
