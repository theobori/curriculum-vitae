\let\nofiles\relax
\documentclass[margin,line]{../res}
\usepackage{hyperref}
\usepackage[utf8]{inputenc}
\usepackage[T1]{fontenc}
\usepackage{microtype}
\usepackage{fourier-orns}
\usepackage{amsmath,amssymb}
\usepackage{lastpage}
\usepackage{fancyhdr}
\usepackage{etaremune}
\usepackage[normalem]{ulem}
\usepackage[style=iso]{datetime2}

\oddsidemargin -.5in
\evensidemargin -.5in
\voffset -25pt
\headsep 25pt
\textwidth=6.0in
\textheight=8.9in
\itemsep=0in
\parsep=0in

% Headings
\pagestyle{fancy}
\lhead{Théo BORI --- Curriculum Vitae}
\chead{}
\rhead{\thepage\ sur \pageref*{LastPage}}
\lfoot{}
\cfoot{}
\rfoot{}
\renewcommand{\headrulewidth}{0.4pt}

% Give hyperref some metadata
\hypersetup{pdftitle={Théo BORI — Curriculum Vitae (CV)},
  pdfauthor={Théo BORI},
  pdfsubject={Théo BORI's curriculum vitae (CV)},
  pdfkeywords={informatique, unix, linux, système, ingénierie logicielle,
  SRE, DevOps, sysadmin, algorithmie},
  colorlinks=true,
  linkcolor=blue,
  filecolor=magenta,
  urlcolor=blue
}

\begin{document}

\newcommand{\myname}{Théo BORI}
\newlength{\mynamewidth}
\settowidth{\mynamewidth}{\namefont\myname}

\name{\hspace*{0.5\textwidth}\hspace{-0.5\mynamewidth} \myname \vspace*{.1in}}
% On the first page, have no header.
\thispagestyle{empty}

\begin{resume}

  \section{\sc Informations de contact}
  %\vspace{.05in}
\href{mailto:theobori@disroot.org}{theobori@disroot.org} \hfill
\href{https://github.com/theobori}{github.com/theobori}\\
\href{https://theobori.cafe}{theobori.cafe} \hfill
\href{https://linkedin.com/in/theo-bori}{linkedin.com/in/theo-bori}\\
\null\hfill
\href{https://repology.org/maintainer/theobori@disroot.org}{repology.org/maintainer/theobori@disroot.org}\\


  % Added to improve page breaks
  \vspace{-1em}

  \section{\sc Profil}
  Passionné par l'écosystème UNIX et le monde open-source depuis
  l'adolescence, je contribue régulièrement à des projets
  communautaires. Je valorise le partage de connaissances et la
  documentation technique rigoureuse.

  \section{\sc Scolarité}
  {\bf Epitech}, Montpellier, Paris, France
  \hfill {\it Septembre 2020--Octobre 2025}
  \vspace*{.05in}
  \begin{itemize}
    \item {\it Diplôme d'Expert en Technologie de l'Information}
  \end{itemize}

  {\bf Technological University Dublin}, Dublin, Irlande
  \hfill {\it Octobre 2023--Août 2024}
  \vspace*{.05in}
  \begin{itemize}
    \item {\it Diplôme en Science des Données}
  \end{itemize}

  \section{\sc Expériences professionnelles}
  {\bf Stage Ingénieur DevOps \& Cloud},
  \href{https://www.thalesgroup.com/fr}{Thales AVS}, Mérignac, France
  \hfill {\it Mars 2025--Août 2025}
  \vspace*{.05in}
  \begin{itemize}
    \item Mise en place de GitLab Workspaces dans un cluster
      Kubernetes géré par Azure (AKS).
    \item Création de projets Python qui automatise la vérification
      de la sécurité des
      projets informatiques de la forge logicielle et les aident à
      respecter les bonnes pratiques de l'entreprise. Ces
      projets sont lancés depuis les pipelines CI/CD GitLab, ils
      produisent des rapports de sécurité détaillés avec
      un score et des graphiques intuitifs.
    \item \'Ecriture d'un algorithme d'analyse de syntaxe pour
      traiter des données produites par l'exécution de tests
      fonctionnels et optimisations via du multi-threading.
    \item Utilisation d'agents de programmation IA, dont
      CodeCompanion, pour analyser puis sécuriser le
      code source d'un logiciel d'exécution de tests.
  \end{itemize}

  {\bf Stage Site Reliability Engineer},
  \href{https://www.vinci-autoroutes.com}{VINCI Autoroutes}, Vedène, France
  \hfill {\it Septembre 2024--Mars 2025}
  \vspace*{.05in}
  \begin{itemize}
    \item Création de ressources Azure suivie du déploiement de
      Backstage.io avec AKS.
    \item Intégration des utilisateurs et des groupes existants dans Microsoft
      Entra ID, mise en place de l'authentification avec Azure,
      sécurisation des accès avec un Application Gateway
      suivi du Azure Firewall, puis intégration des services de
      l'usine logicielle à Backstage.io.
    \item Création d'un projet Python qui produit un CLI permettant
      de simplifier l'ajout de projets existants dans Backstage.io.
  \end{itemize}

  {\bf Stage Site Reliability Engineer},
  \href{https://www.vinci-autoroutes.com}{VINCI Autoroutes}, Vedène, France
  \hfill {\it Avril 2023--Juillet 2023}
  \vspace*{.05in}
  \begin{itemize}
    \item Déploiement automatisé des ressources dans Azure, de
      NeuVector (AKS) et de sa configuration avec Terraform. La
      surveillance a été mise en place avec Prometheus de Grafana,
      l'ensemble de la mise en œuvre de la solution
      a été entièrement documenté.
    \item Contributions au code source publique de NeuVector et à celui de
      l'exporteur Prometheus de NeuVector.
    \item Création d'une
      \href{https://github.com/theobori/go-neuvector}{bibliothèque}
      en Go pour interagir avec le Controller NeuVector ainsi qu'un
      \href{https://registry.terraform.io/providers/theobori/neuvector/latest}{provider
      Terraform} en Go qui permet de gérer le cycle de vie des objets NeuVector.
  \end{itemize}

  \section{\sc Expériences personnelles notables}

  {\bf Site web et services}
  \hfill {\it Septembre 2022--Aujourd'hui}
  \vspace*{.05in}
  \begin{itemize}
    \item \href{https://theobori.cafe}{Site web} sur lequel je
      partage les choses que je trouve intéressantes.
    \item Mise à disposition de
      \href{https://services.theobori.cafe}{services} gratuits et
      ouverts à toutes et à tous. Ils sont tous conteneurisés grâce
      à Docker et accessibles derrière NGINX, le déploiement est
      automatisé avec un Playbook Ansible et les enregistrements DNS
      sont gérés avec un projet Terraform. Prometheus et Grafana sont
      utilisés pour surveiller l'état des services et du système et des
      sauvegardes automatiques sont réalisées chaque semaine.
      L'authentification aux services est gérée par un serveur
      OpenLDAP et des mécanismes de blocage ont été mis en place pour
      les tentatives d'authentification malveillantes.  Toute
      la documentation technique et
      les statistiques utilisateurs sont
      \href{https://services.theobori.cafe/documentation}{disponibles
      publiquement}.
    \item Maintenance depuis plus d'un an d'une
      \href{https://zettel.theobori.cafe}{base de connaissances} en
      ligne organisée qui met à disposition toutes mes notes.
  \end{itemize}

  {\bf Paquetage d'applications}
  \hfill {\it Mars 2024--Aujourd'hui}
  \vspace*{.05in}
  \begin{itemize}
    \item
      \href{https://repology.org/maintainer/theobori@disroot.org}{Contributions}
      régulières à la collection d'expressions officielles de Nix(OS)
      (plus d'une centaine en 2025) et
      \href{https://github.com/theobori/nix-teeworlds}{modularisation}
      de service.
    \item \href{https://github.com/theobori/openbsd-ports}{Portages}
      d'applications et de jeux compatibles avec X11 pour OpenBSD.
  \end{itemize}

  {\bf Algorithmie et structures de données}
  \hfill {\it Décembre 2023--Aujourd'hui}
  \vspace*{.05in}
  \begin{itemize}
    \item Bases théoriques et pratiques solides en algorithmie et
      structures de données.
    \item \href{https://leetcode.com/u/theobori/}{Résolution} de
      problèmes algorithmiques sur différentes plateformes dédiées et
      complétions de
      l'\href{https://github.com/theobori/advent-of-code}{Advent Of Code}.
  \end{itemize}

  {\bf Filtrage de paquets avec Linux}
  \hfill {\it Janvier 2024--Novembre 2024}
  \vspace*{.05in}
  \begin{itemize}
    \item \'Ecriture d'un
      \href{https://github.com/theobori/tinyfilter}{projet XDP} en C pour
      filtrer les paquets dans l'espace noyau. Le but est d'avoir un
      filtreur de paquets haute performance
      chargé dans l'espace noyau, avec la possibilité de gérer des
      règles de filtrage personnalisées pour différentes couches
      réseau depuis l'espace utilisateur.
  \end{itemize}

  {\bf Interpréteurs Lox}
  \hfill {\it Mars 2024--Novembre 2024}
  \vspace*{.05in}
  \begin{itemize}
    \item Réalisation d'interpréteurs du langage Lox en partant de
      zéro avec différentes méthodes d'analyse syntaxique et
      d'évaluation de code, dont un
      \href{https://github.com/theobori/tinylox}{interpréteur
      Tree-Walk} en Python et une
      \href{https://github.com/theobori/lox-virtual-machine}{machine
      virtuelle de bytecodes} en C.
  \end{itemize}

  \section{\sc Langues}
  \begingroup
  \raggedright
  {\bf Français}, langue maternelle\\
  \vspace*{.15in}
  {\bf Anglais}, parlé, lu et écrit
  \endgroup

\end{resume}
\end{document}
