\let\nofiles\relax
\documentclass[margin,line]{../res}
\usepackage{hyperref}
\usepackage[utf8]{inputenc}
\usepackage[T1]{fontenc}
\usepackage{microtype}
\usepackage{fourier-orns}
\usepackage{amsmath,amssymb}
\usepackage{lastpage}
\usepackage{fancyhdr}
\usepackage{etaremune}
\usepackage[normalem]{ulem}
\usepackage[style=iso]{datetime2}

\oddsidemargin -.5in
\evensidemargin -.5in
\voffset -25pt
\headsep 25pt
\textwidth=6.0in
\textheight=8.9in
\itemsep=0in
\parsep=0in

% Headings
\pagestyle{fancy}
\lhead{Théo BORI --- Curriculum Vitae}
\chead{}
\rhead{\thepage\ of \pageref*{LastPage}}
\lfoot{}
\cfoot{}
\rfoot{}
\renewcommand{\headrulewidth}{0.4pt}

% Give hyperref some metadata
\hypersetup{pdftitle={Théo BORI – Curriculum Vitae (CV)},
  pdfauthor={Théo BORI},
  pdfsubject={Théo BORI's curriculum vitae (CV)},
  pdfkeywords={computer science, unix, linux, system, software engineering,
  SRE, DevOps, sysadmin, algorithms},
  colorlinks=true,
  linkcolor=blue,
  filecolor=magenta,
  urlcolor=blue
}

\begin{document}

\newcommand{\myname}{Théo BORI}
\newlength{\mynamewidth}
\settowidth{\mynamewidth}{\namefont\myname}

\name{\hspace*{0.5\textwidth}\hspace{-0.5\mynamewidth} \myname \vspace*{.1in}}
% On the first page, have no header.
\thispagestyle{empty}

\begin{resume}

  \section{\sc Contact Information}
  %\vspace{.05in}
\href{mailto:theobori@disroot.org}{theobori@disroot.org} \hfill
\href{https://github.com/theobori}{github.com/theobori}\\
\href{https://theobori.cafe}{theobori.cafe} \hfill
\href{https://linkedin.com/in/theo-bori}{linkedin.com/in/theo-bori}\\
\null\hfill
\href{https://repology.org/maintainer/theobori@disroot.org}{repology.org/maintainer/theobori@disroot.org}\\


  % Added to improve page breaks
  \vspace{-1em}

  \section{\sc Profile}
  Passionate about the UNIX ecosystem and the open-source world since
  my teenage years, I regularly contribute to community projects. I value
  knowledge sharing and rigorous technical documentation.

  \section{\sc Education}
  {\bf Epitech}, Montpellier, Paris, France
  \hfill {\it September 2020--October 2025}
  \vspace*{.05in}
  \begin{itemize}
    \item {\it Expert Degree in Information Technology}
  \end{itemize}

  {\bf Technological University Dublin}, Dublin, Ireland
  \hfill {\it October 2023--August 2024}
  \vspace*{.05in}
  \begin{itemize}
    \item {\it Degree in Data Science}
  \end{itemize}

  \section{\sc Professional Experience}
  {\bf DevOps \& Cloud Engineer Intern},
  \href{https://www.thalesgroup.com/fr}{Thales AVS}, Mérignac, France
  \hfill {\it March 2025--August 2025}
  \vspace*{.05in}
  \begin{itemize}
    \item Deployment of GitLab Workspaces in a Kubernetes cluster
      managed by Azure (AKS).
    \item Creation of Python projects that automate security verification
      of IT projects in the software forge and help them comply with
      company best practices. These projects are launched from GitLab
      CI/CD pipelines and produce detailed security reports with
      scores and intuitive graphics.
    \item Development of a syntax analysis algorithm to process
      data produced by functional test execution and optimizations
      through multi-threading.
    \item Use of AI programming agents, including CodeCompanion,
      to analyze and secure the source code of test execution software.
  \end{itemize}

  {\bf Site Reliability Engineer Intern},
  \href{https://www.vinci-autoroutes.com}{VINCI Autoroutes}, Vedène, France
  \hfill {\it September 2024--March 2025}
  \vspace*{.05in}
  \begin{itemize}
    \item Creation of Azure resources followed by deployment of
      Backstage.io with AKS.
    \item Integration of existing users and groups in Microsoft
      Entra ID, implementation of authentication with Azure, securing
      access with an Application Gateway
      followed by Azure Firewall, then integration of software factory
      services into Backstage.io.
    \item Creation of a Python project that produces a CLI to
      simplify the addition of existing projects in Backstage.io.
  \end{itemize}

  {\bf Site Reliability Engineer Intern},
  \href{https://www.vinci-autoroutes.com}{VINCI Autoroutes}, Vedène, France
  \hfill {\it April 2023--July 2023}
  \vspace*{.05in}
  \begin{itemize}
    \item Automated deployment of resources in Azure, NeuVector (AKS)
      and its configuration with Terraform. Monitoring was set up
      with Grafana's Prometheus, and the entire solution implementation
      was fully documented.
    \item Contributions to NeuVector's public source code and to
      NeuVector's Prometheus exporter.
    \item Creation of a
      \href{https://github.com/theobori/go-neuvector}{library}
      in Go to interact with the NeuVector Controller as well as a
      \href{https://registry.terraform.io/providers/theobori/neuvector/latest}{Terraform
      provider} in Go that manages the lifecycle of NeuVector objects.
  \end{itemize}

  \section{\sc Notable\\ Personal\\ Projects}

  {\bf Website and Services}
  \hfill {\it September 2022--Present}
  \vspace*{.05in}
  \begin{itemize}
    \item \href{https://theobori.cafe}{Website} where I share
      things I find interesting.
    \item Provision of
      \href{https://services.theobori.cafe}{free services} open to
      everyone. They are all containerized with Docker and accessible
      behind NGINX, deployment is automated with an Ansible Playbook
      and DNS records are managed with a Terraform project. Prometheus
      and Grafana are used to monitor service and system status and
      automatic backups are performed weekly. Service authentication
      is managed by an OpenLDAP server and blocking mechanisms have
      been implemented for malicious authentication attempts. All
      technical documentation and user statistics are
      \href{https://services.theobori.cafe/documentation}{publicly
      available}.
    \item Maintenance for over a year of an organized online
      \href{https://zettel.theobori.cafe}{knowledge base} that provides
      all my notes.
  \end{itemize}

  {\bf Application Packaging}
  \hfill {\it March 2024--Present}
  \vspace*{.05in}
  \begin{itemize}
    \item Regular
      \href{https://repology.org/maintainer/theobori@disroot.org}{contributions}
      to the official Nix(OS) expression collection (over a hundred in
      2025) and service
      \href{https://github.com/theobori/nix-teeworlds}{modularization}.
    \item \href{https://github.com/theobori/openbsd-ports}{Porting} of
      X11-compatible applications and games for OpenBSD.
    \item \href{https://github.com/theobori/9ports}{Portages} of Plan
      9 programs for UNIX.
  \end{itemize}

  {\bf Algorithms and Data Structures}
  \hfill {\it December 2023--Present}
  \vspace*{.05in}
  \begin{itemize}
    \item Strong theoretical and practical foundations in algorithms
      and data structures.
    \item \href{https://leetcode.com/u/theobori/}{Solving} algorithmic
      problems on various dedicated platforms and completing the
      \href{https://github.com/theobori/advent-of-code}{Advent Of Code}.
  \end{itemize}

  {\bf Packet Filtering with Linux}
  \hfill {\it January 2024--November 2024}
  \vspace*{.05in}
  \begin{itemize}
    \item Development of an
      \href{https://github.com/theobori/tinyfilter}{XDP project} in C to
      filter packets in kernel space. The goal is to have a
      high-performance packet filter loaded in kernel space, with the
      ability to manage custom filtering rules for different network
      layers from user space.
  \end{itemize}

  {\bf Lox Interpreters}
  \hfill {\it March 2024--November 2024}
  \vspace*{.05in}
  \begin{itemize}
    \item Implementation of interpreters for the Lox language from
      scratch with different parsing and code evaluation methods,
      including a
      \href{https://github.com/theobori/tinylox}{Tree-Walk interpreter}
      in Python and a
      \href{https://github.com/theobori/lox-virtual-machine}{bytecode
      virtual machine} in C.
  \end{itemize}

  \section{\sc Languages}
  \begingroup
  \raggedright
  {\bf French}, native language\\
  \vspace*{.15in}
  {\bf English}, spoken, read and written
  \endgroup

\end{resume}
\end{document}
